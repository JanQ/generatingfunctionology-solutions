
\section{Analytic and Asymptotic Methods}
\begin{exercise}
    Use the LIF to show that the (infinite) binomial coefficient sum
    \[
        \xi = \sum_s \binom{sL+1}{s} \frac{A^{-sL-1}}{(sL+1)},
    \]
    for $A>1$ and integer $L>0$, satisfies $\xi^L - A\xi + 1 =0$.
\end{exercise}
\begin{solution}
    Use Theorem~\ref{thm:lif} with $t=\frac{1}{A}$, $\phi(u) = 1+u^L$ and $f(u) = u$ to find (the associated functional equation is $u = \frac{1}{A}(1+u^L)$ which is satisfied by $\xi$):
    \begin{align*}
      \xi &= \bsum[n]\frac{1}{A^n} \frac{1}{n}\coeff{\xi^{n-1}} (1+\xi^L)^n \\
        \estepalign{\eqref{eq:binom_num}} \bsum[n] \frac{1}{A^n}\frac{1}{n} \coeff{\xi^{n-1}} \infsum[k] \binom{n}{k}\xi^{Lk} \\
        &= \infsum[k] \binom{kL+1}{k} \frac{A^{-kL-1}}{kL+1}
    \end{align*}
    where we set $n-1=Lk$ in the last equality.
\end{solution}

\begin{exercise}
    \label{ex:5-2}
    The Legendre polynomials $\{P_n(x)\}$ are generated by
    \[
        \frac{1}{\sqrt{1-2xt+t^2}} = \sum_{n\geq0} P_n(x)t^n
    \]
    Let $x$ be a fixed complex number that lies outside the real interval $[-1,1]$, and let $\tau$ denote one of the two roots of the equation $\tau^2-2x\tau+1=0$ which is $>1$ in absolute value. Use the method of Darboux to show that, as $n\to \infty$,
    \[
        P_n(x)\sim \frac{\tau^{n+1}}{\sqrt{n\pi(\tau^2-1)}}
    \]
\end{exercise}
\begin{solution}
    Let $\sigma$ denote the other root of $t^2 - 2xt + 1$. Note that:
    \[
        (\tau - t)(\sigma - t) = \tau\sigma -t(\tau + \sigma) + t^2 = t^2 - 2xt + 1
    \]
    and therefore $\sigma = \frac{1}{\tau}$ where $\sigma < 1$ in absolute value. Scaling the generating function for the Legendre polynomials by $\sigma$ gives:
    \[
        \bsum[n] P_n(x) (\sigma t)^n = \frac{1}{\sqrt{(\tau - \sigma t) (\sigma - \sigma t)}} = \frac{1}{\sqrt{(\sigma^{-1} - \sigma t)\sigma(1-t)}} = \frac{(1-t)^{-1/2}}{\sqrt{1-\sigma^2t}} 
    \]
    where $v(t) = \frac{1}{\sqrt{1-\sigma^2t}}$ is analytic in some disk $|t| < 1+\eta$ (because $|\sigma^2| < 1$) such that the requirements for the method of Darboux are met with $\beta = -\frac{1}{2}$. Now apply Theorem~\ref{thm:darboux} with $m=0$ to obtain:
    \[
        P_n(x)\sigma^n = v(1) \binom{n-\beta-1}{n} + \mathcal{O}(n^{-\beta-2}) = \frac{\tau}{\sqrt{\tau^2-1}}\binom{n-1/2}{n} + \mathcal{O}(n^{-3/2})
    \]
    Because $\binom{n-1/2}{n} = \frac{n^{-1/2}}{\sqrt{\pi}}\left(1+ \mathcal{O}(n^{-1})\right)$, see \eqref{eq:binom_asymp}, we then have:
    \[
        P_n(x)\sigma^n = \frac{\tau}{\sqrt{\tau^2-1}} \frac{1}{\sqrt{n\pi}} + \mathcal{O}(n^{-3/2})
    \]
    Moving $\sigma^n = \frac{1}{\tau^n}$ to the right side gives the desired result:
    \[
        P_n(x)\sim \frac{\tau^{n+1}}{\sqrt{n\pi(\tau^2-1)}}
    \]
\end{solution}

\begin{exercise}
    \label{ex:5-3}
    If $u=u(t)$ satisfies $u=t\phi(u)$ and $n\geq0$, show that 
    \[
        \coeff{u^n}\{\phi(u)\}^n = \coeff{t^n}\left\{\frac{tu'(t)}{u(t)} \right\} = \coeff{t^n}\frac{1}{(1-t\phi'(u(t)))}
    \]
\end{exercise}
\begin{solution}
    For the equality, choose $f(u)$ such that $f'(u) = \frac{1}{\phi(u)}$. Then from Theorem~\ref{thm:lif}, we have:
    \[
        \coeff{t^n} f(u(t)) = \frac{1}{n}\coeff{u^{n-1}} \phi(u)^{n-1}
    \]
    The left-hand side can be rewritten as follows:
    \[
        \coeff{t^n} f(u(t)) \estep{\eqref{eq:xD}} \frac{1}{n} \coeff{t^{n-1}} D_t f(u(t)) = \frac{1}{n} \coeff{t^{n-1}} \frac{1}{\phi(u)} u'(t) = \frac{1}{n} \coeff{t^{n-1}} \frac{tu'(t)}{u(t)}
    \]
    where the chain rule was used in the second equality and $u=t\phi(u)$ in the last equality. We therefore have proven that:
    \[
        \coeff{t^{n-1}} \frac{tu'(t)}{u(t)} = \coeff{u^{n-1}} \phi(u)^{n-1}
    \]
    To prove the second equality, differentiating $u(t) = t\phi(u(t))$ gives:
    \[
        u'(t) = \phi(u(t)) + t\phi'(u(t))u'(t) \Longleftrightarrow u'(t) = \frac{\phi(u(t))}{(1-t\phi'(u(t)))}
    \]
    Filling this in the left-hand side of the desired equality yields the result:
    \[
        \coeff{t^n}\left\{\frac{tu'(t)}{u(t)} \right\} = \coeff{t^n} \left\{\frac{t\phi(u(t))}{u(t)(1-t\phi'(u(t)))} \right\} = \coeff{t^n}\frac{1}{(1-t\phi'(u(t)))}
    \]
\end{solution}

\begin{exercise}
    Define, for all $n\geq 0$, $\gamma_n = \coeff{x^n}(1+x+x^2)^n$.
    \begin{enumerate}[label=(\alph*)]
        \item Use the result of Exercise \ref{ex:5-3} above to prove that for $n\geq0$,
        \[
            \gamma_n = \coeff{x^n}\left\{\frac{1}{\sqrt{1-2x-3x^2}}\right\}
        \]
        \item Show that, using the notation of Exercise \ref{ex:5-2} above,
        \[
            \gamma_n = \left(\sqrt{3}/i\right)^n P_n(i/\sqrt{3}),
        \]
        and so obtain the asymptotic behavior of the sequence $\{\gamma_n\}$ for large $n$.
    \end{enumerate}
\end{exercise}
\begin{solution}
    \begin{enumerate}[label=(\alph*)]
        \item Let $\phi(u) = (1+u+u^2)$, then the result from Exercise \ref{ex:5-3} tells us that:
        \[
            \coeff{u^n} (1+u+u^2)^n = \coeff{t^n} \frac{1}{(1-t(1+2u))}
        \]
        We also know that $u=t\phi(u)  = t(1+u+u^2)$. Solving this quadratic in $u$ (the negative solution is chosen to ensure $u=0$ at $t=0$):
        \[
            u = \frac{-(t-1) - \sqrt{-3t^2 -2t+1}}{2t}
        \]
        Filling this in in the identity:
        \[
            \coeff{u^n} (1+u+u^2)^n = \coeff{t^n} \frac{1}{\sqrt{1-2t-3t^2}}
        \]
        Substituting $u=x$ and $t=x$ gives the desired identity:
        \[
            \gamma_n = \coeff{x^n} (1+x+x^2)^n =  \coeff{x^n}\left\{\frac{1}{\sqrt{1-2x-3x^2}}\right\}
        \]
        \item Let $x=\frac{i}{\sqrt{3}}$ in the generating function of the Legendre polynomials to find:
        \[
            \bsum[n] P_n(i/\sqrt{3})t^n = \frac{1}{\sqrt{1-\frac{2ti}{\sqrt{3}}+t^2}}
        \]
        Scaling $t$ by $\frac{\sqrt{3}}{i}$:
        \[
            \bsum[n] P_n(i/\sqrt{3})\left(\sqrt{3}/i\right)^nt^n = \frac{1}{\sqrt{1-2t-3t^2}}
        \]
        Extracting the coefficient of $t^n$ from both sides:
        \[
            P_n(i/\sqrt{3})\left(\sqrt{3}/i\right)^n = \coeff{t^n} \frac{1}{\sqrt{1-2t-3t^2}} = \gamma_n
        \]
        The root with absolute value $>1$ of $t^2 - 2ti/\sqrt{3} + 1$ is $\tau = i\sqrt{3}$, the asymptotic behavior of $\gamma_n$ by Exercise~\ref{ex:5-2} is therefore:
        \[
            \gamma_n \sim \frac{\tau^{n+1}}{\sqrt{n\pi(\tau^2-1)}} \left(\frac{\sqrt{3}}{i}\right)^n = \frac{3^n\sqrt{3}}{2\sqrt{n\pi}}
        \]
    \end{enumerate}
\end{solution}

\begin{exercise}
    Define, for integer $p\geq 3$,
    \[
        S_p(n) = \sum_{k=0}^n \binom{pn}{k} \quad (n\geq 0)
    \]
    \begin{enumerate}[label=(\alph*)]
        \item Exhibit $S_p(n)$ as $\coeff{x^n}$ in a certain ordinary power series, which (alas!) itself depends on $n$.
        \item Nevertheless, use the LIF (backwards) to show that 
        \[
            \sum_n S_p(n) x^n (1+x)^{-pn-1} = \frac{1}{(1-x)(1-(p-1)x)}
        \]
        \item Deduce from part (b) that the $\{S_p(n)\}$ satisfy the recurrence 
        \[
            \sum_k (-1)^k \binom{pn-(p-1)k}{k} S_p(n-k) = \frac{(p-1)^{n+1} - 1}{p-2} \quad (n\geq 0)
        \]
        \item If $F(u) = \sum_{n\geq 0}S_p(n)u^n$, let
        \[
            x= \frac{1}{(p-1)} - \epsilon
        \]
        in part (b) to show that 
        \[
            F\left(\frac{(p-1)^{p-1}}{p^p}\left\{1- \frac{(p-1)^3}{2p}\epsilon^2 + \cdots\right\}\right) = \frac{p}{(p-1)(p-2)\epsilon} + \mathcal{O}(1)
        \]
        as $\epsilon\to0$.
        \item If 
        \[
            g(x) = F\left(\frac{(p-1)^{p-1}}{p^p}x\right)
        \]
        then show that 
        \[
            g(x) = \frac{1}{(p-2)}\sqrt{\binom{p}{2}}\frac{1}{\sqrt{1-x}} + \mathcal{O}(1)
        \]
        \item Use Darboux's method to show that, as $n\to \infty$,
        \[
            S_p(n) \sim \frac{1}{(p-2)} \sqrt{\frac{\binom{p}{2}}{n\pi}} \left(\frac{p^p}{(p-1)^{p-1}}\right)^n
        \]
        \item From part (b) show that 
        \[
            \sum_{n\geq0} S_3(n)\left(\frac{4u^2}{27}\right)^n = \frac{u}{u-2\sin\left(\frac{1}{3}\sin^{-1} u\right)} - \frac{2u}{2u-3\sin\left(\frac{1}{3}\sin^{-1} u\right)}
        \]
    \end{enumerate}
\end{exercise}
\begin{solution}
    \begin{enumerate}[label=(\alph*)]
        \item $S_p(n)$ is a prefix sum of binomial coefficients. We then immediately find that:
        \[
            S_p(n) = \coeff{x^n} \frac{1}{1-x} (1+x)^{pn}
        \]
        \item \hypertarget{eq:ch5:5:b}{} Let $\phi(u) = (1+u)^p$ and choose $f(u)$ such that $f'(u) = \frac{1}{(1-u)(1+u)^p}$, then from Theorem~\ref{thm:lif}:
        \[
            \frac{1}{n}\coeff{u^{n-1}} \left(\frac{(1+u)^{pn}}{(1-u)(1+u)^p}\right) = \coeff{t^n} (f(u(t)))
        \]
        Shifting the index $n$ by $1$:
        \[
            \frac{1}{n+1}\coeff{u^{n}} \left(\frac{(1+u)^{pn}}{(1-u)}\right) = \coeff{t^{n+1}} (f(u(t)))
        \]
        On the left side, the desired term is found:
        \[
            \frac{1}{n+1} S_p(n) = \coeff{t^{n+1}} (f(u(t)))
        \]
        Because $\coeff{t^{n+1}} f(t) \estep{\eqref{eq:xD}} \frac{1}{n+1}\coeff{t^n} f'(t)$ and applying the chain rule:
        \[
            \frac{1}{n+1} S_p(n) = \frac{1}{n+1}\coeff{t^{n}} (f'(u(t)) u'(t))
        \]
        $f'(u)$ is known and from $u(t) = t\phi(u(t)) = t(1+u(t))^p$ we obtain:
        \[
            u'(t) = (1+u(t))^p + tp(1+u(t))^{p-1}u'(t)
        \]
        or:
        \[
            u'(t) = \frac{(1+u(t))^p}{1-tp(1+u(t))^{p-1}} = \frac{(1+u)^{p+1}}{(1-(p-1)u)}
        \]
        where we used $u = t(1+u)^p$ in the last equality in the denominator. Filling this expression back in the identity:
        \begin{align*}
            S_p(n) &= \coeff{t^n} \frac{1}{(1-u)(1+u)^p} \frac{(1+u)^{p+1}}{(1-(p-1)u)} \\
            &= \coeff{t^n}\frac{(1+u)}{(1-u)(1-(p-1)u)}
        \end{align*}
        Now let $x=u$ and therefore $t = \frac{x}{(1+x)^p}$, multiply both sides by $t^n$ and sum over $n$:
        \[
            \bsum S_p(n)x^n (1+x)^{-pn} = \frac{(1+x)}{(1-x)(1-(p-1)x)}
        \]
        Diving both sides by $(1+x)$ gives the desired result:
        \[
            \bsum S_p(n) x^n (1+x)^{-pn-1} = \frac{1}{(1-x)(1-(p-1)x)}
        \]
        \item Equating coefficients of $x^n$ on both sides in the previous result, using \eqref{eq:binom_denom} on the left-hand side and \eqref{eq:power_geom} on the right-hand side leads to:
        \begin{align*}
            \coeff{x^{n}} \bsum[\tilde{n}] S_p(\tilde{n})x^{\tilde{n}} \infsum[k] \binom{k + p\tilde{n}}{k}(-x)^k &= \coeff{x^n} \bsum[k] x^k \bsum[l] ((p-1)x)^l \\
            \Longleftrightarrow \coeff{x^n} \infsum[k] \bsum[\tilde{n}] S_p(\tilde{n}) \binom{k+p\tilde{n}}{k} (-1)^k x^{k + \tilde{n}} &= \coeff{x^n} \bsum[l]x^l \sum_{k=0}^l (p-1)^k\\
            \Longleftrightarrow \infsum[k] S_p(n-k)\binom{p(n-k) + k}{k}(-1)^k &= \frac{(p-1)^{n+1} - 1}{p-2}
        \end{align*}
        \item \hypertarget{eq:ch5:5:d}{} Let $F(u) = \bsum[n]S_p(n)u^n$ and fill in $x=\frac{1}{p-1}-\epsilon$ on both sides in the result of \hyperlink{eq:ch5:5:b}{5.5(b)} to obtain:
        \[
            F\left(\frac{\frac{1}{p-1}-\epsilon}{\left(1+\frac{1}{p-1}-\epsilon\right)^p}\right) = \frac{1+\frac{1}{p-1}-\epsilon}{\left(1-\frac{1}{p-1}-\epsilon\right)\left(1-(p-1)\frac{1}{p-1} + (p-1)\epsilon\right)}
        \]
        Working out the right-hand side:
        \begin{align*}
            \textnormal{r.h.s.} &= \frac{\frac{(p-1 + 1 - (p-1)\epsilon)}{p-1}}{((p-1) - 1 - \epsilon(p-1)) \epsilon} \\
            &= \frac{p - (p-1)\epsilon}{(p-2)(p-1)\epsilon - (p-1)^2\epsilon^2} \\
            &= \frac{p}{(p-2)(p-1)\epsilon - (p-1)^2\epsilon^2} - \frac{\epsilon}{(p-2)\epsilon - (p-1)\epsilon^2}
        \end{align*}
        The second term approaches the constant $\frac{-1}{p-2}$ as $\epsilon \to 0$ and in the first term $(p-1)^2\epsilon^2$ is negligible with respect to the first term when $\epsilon \to 0$. We therefore find:
        \[
            \textnormal{r.h.s.} = \frac{p}{(p-2)(p-1)\epsilon} + \mathcal{O}(1)
        \]
        The argument of $F$ on the left-hand side is:
        \begin{align*}
            \frac{\frac{1}{p-1}-\epsilon}{\left(1+\frac{1}{p-1}-\epsilon\right)^p} &= \frac{1-\epsilon(p-1)}{(p-1)} \frac{(p-1)^p}{(p-1 + 1 - \epsilon(p-1))^p} \\
            &= \frac{(p-1)^{p-1}}{p^p} \frac{1-\epsilon(p-1)}{(1-\epsilon(p-1)/p)^p}
        \end{align*}
        To evaluate this last term, we use the binomial series expansion \eqref{eq:binom_denom}:
        \[
            \frac{1}{(1-\epsilon(p-1)/p)^p} = \left(1 + \binom{p}{1}\frac{\epsilon(p-1)}{p}+ \binom{p+1}{2}\frac{\epsilon^2(p-1)^2}{p^2}+\cdots\right)
        \]
        Because $\binom{p}{1} = p$ and $\binom{p+1}{2} = \frac{(p+1)p}{2}$:
        \[
            \frac{1}{(1-\epsilon(p-1)/p)^p} = \left(1 + \epsilon(p-1)+\frac{(p+1)\epsilon^2(p-1)^2}{2p}+\cdots\right)
        \]
        Multiplying by $(1-\epsilon(p-1))$ gives (omitting all the terms which are $\mathcal{O}(\epsilon^3)$):
        \[
            \frac{1-\epsilon(p-1)}{(1-\epsilon(p-1)/p)^p} = 1 - \frac{(p-1)^3}{2p}\epsilon^2 + \cdots
        \]
        such that the argument of $F$ on the left-hand side is:
        \[
            \frac{\frac{1}{p-1}-\epsilon}{\left(1+\frac{1}{p-1}-\epsilon\right)^p} = \frac{(p-1)^{p-1}}{p^p}\left\{1- \frac{(p-1)^3}{2p}\epsilon^2 + \cdots\right\}
        \]
        Combine left and right side to obtain the result:
        \[
            F\left(\frac{(p-1)^{p-1}}{p^p}\left\{1- \frac{(p-1)^3}{2p}\epsilon^2 + \cdots\right\}\right) = \frac{p}{(p-1)(p-2)\epsilon} + \mathcal{O}(1)
        \]
        \item Note that
        \[
            1 - \frac{(p-1)^3}{2p} \epsilon^2 = x \Longleftrightarrow  \epsilon = \sqrt{\frac{2p(1-x)}{(p-1)^3}}
        \]
        Using the result of \hyperlink{eq:ch5:5:d}{5.5(d)}:
        \begin{align*}
            F\left(\frac{(p-1)^{p-1}}{p^p} x\right) &= \frac{p}{(p-1)(p-2) \frac{\sqrt{2p}\sqrt{1-x}}{(p-1)\sqrt{p-1}}} + \mathcal{O}(1) \\
            &= \frac{1}{(p-2)}\sqrt{\frac{p(p-1)}{2}} \frac{1}{\sqrt{1-x}} + \mathcal{O}(1) \\
            &= \frac{1}{(p-2)} \sqrt{\binom{p}{2}} \frac{1}{\sqrt{1-x}}+ \mathcal{O}(1)
        \end{align*}
        \item Applying Theorem~\ref{thm:darboux} on $g(x)$ with $v(x) = 1$, $\beta=-\frac{1}{2}$ and $m=0$ (note that the term in front is independent of $x$):
        \begin{align*}
            \coeff{x^n} g(x) &= \frac{1}{(p-2)} \sqrt{\binom{p}{2}} \binom{n-1/2}{n} + \mathcal{O}(1) \\
            \estepalign{\eqref{eq:binom_asymp}} \frac{1}{(p-2)} \sqrt{\binom{p}{2}} \frac{1}{\sqrt{n\pi}} (1 - \mathcal{O}(n^{-1})) + \mathcal{O}(1)
        \end{align*}
        Now, using the definitions of $g(x)$ and $F(x)$:
        \[
            g(x) = F\left(\frac{(p-1)^{p-1}}{p^p}x\right) = \bsum[n] S_p(n) \left(\frac{(p-1)^{p-1}}{p^p}x\right)^n
        \]
        The coefficient of $x^n$ is:
        \[
            \coeff{x^n}g(x) = S_p(n) \left(\frac{(p-1)^{p-1}}{p^p}\right)^n
        \]
        from which the result follows:
        \[
            S_p(n) \sim \frac{1}{(p-2)} \sqrt{\frac{\binom{p}{2}}{n\pi}}\left(\frac{p^p}{(p-1)^{p-1}}\right)^n
        \]
        \item From part \hyperlink{eq:ch5:5:b}{5.5(b)}, we have that:
        \[
            \bsum[n] S_3(n) \left(\frac{x}{(1+x)^3}\right)^n = \frac{1+x}{(1-x)(1-2x)}
        \]
        Now let $x = \frac{-u + 3 \sin\left(\frac{1}{3}\sin^{-1}u \right)}{u}$. Then on the left-hand side we have:
        \[
            \frac{x}{(1+x)^3} = \frac{-u + 3 \sin\left(\frac{1}{3}\sin^{-1}u\right)}{\frac{1}{u^2}\left(3\sin\left(\frac{1}{3}\sin^{-1}u\right)\right)^3}
        \]
        Using the identity $\sin^3 x = \frac{3\sin x - \sin(3x)}{4}$:
        \[
            \frac{x}{(1+x)^3} = \frac{u^2\left(-u + 3 \sin\left(\frac{1}{3}\sin^{-1}u\right)\right)}{\frac{27}{4}\left(3\sin\left(\frac{1}{3}\sin^{-1}u\right) - u\right)} = \frac{4u^2}{27}
        \]
        For the right-hand side we obtain:
        \begin{align*}
            \frac{1+x}{(1-x)(1-2x)} &= \frac{3u\sin\left(\frac{1}{3}\sin^{-1}u\right)}{\left(2u - 3\sin\left(\frac{1}{3}\sin^{-1}u\right)\right)\left(3u - 6\sin\left(\frac{1}{3}\sin^{-1}u\right)\right)} \\
            &= \frac{u\sin\left(\frac{1}{3}\sin^{-1}u\right)}{\left(u - 2\sin\left(\frac{1}{3}\sin^{-1}u\right)\right)\left(2u - 3\sin\left(\frac{1}{3}\sin^{-1}u\right)\right)} \\
            &= \frac{u}{u-2\sin\left(\frac{1}{3}\sin^{-1} u\right)} - \frac{2u}{2u-3\sin\left(\frac{1}{3}\sin^{-1} u\right)}
        \end{align*}
        where the last equality is easily checked in the reverse direction. The desired identity then follows:
        \[
            \bsum[n] S_3(n)\left(\frac{4u^2}{27}\right)^n = \frac{u}{u-2\sin\left(\frac{1}{3}\sin^{-1} u\right)} - \frac{2u}{2u-3\sin\left(\frac{1}{3}\sin^{-1} u\right)}
        \]
    \end{enumerate}
\end{solution}

\begin{exercise}
    \label{ex:5-6}
    Under what additional conditions on a polynomial $P$ with nonnegative real coefficients will there exists an $N$ such that for all $n>N$ we have $\coeff{z^n}e^{P(z)}>0$?
\end{exercise}
\begin{solution}
    The coefficients of $\coeff{z^n}e^{P(z)}$ are given by sums of coefficients of $P(z)^k$ for $k\in \mathbb{N}$, but the powers in $P(z)^k$ with nonzero coefficients are the powers which can be formed with $k$ terms of the powers in $P(z)$ with nonzero coefficients. Now, by Schur's theorem, for sufficiently large $n$ every number is representable in such a way if the numbers are coprime. Therefore, let $x^{a_1}$, $x^{a_2}$, \ldots, $x^{a_m}$ denote the powers with positive coefficients, then the additional needed condition is:
    \[
        \gcd(a_1, a_2, \ldots, a_m) = 1
    \]
\end{solution}

\begin{exercise}
    \label{ex:5-7}
    Find the asymptotic behavior (main term) of $(1+\epsilon_n)^n$ if 
    \begin{enumerate}[label=(\alph*)]
        \item $\epsilon_n = n^a \qquad (0 < a < 1)$
        \item $\epsilon_n = n^{-a} \qquad (0 < a < 1)$
        \item $\epsilon_n = n^{-a} \log n\qquad (1<a<2)$
    \end{enumerate}
\end{exercise}
\begin{solution}
    \begin{enumerate}[label=(\alph*)]
        \item \begin{align*}
            (1+n^a)^n &= n^{an}(n^{-a} + 1)^n \\
            &= n^{an} \exp(n \log(1+n^{-a})) \\
            \estepalign{\eqref{eq:power_loggeom}} n^{an} \exp\left\{n \left(n^{-a} - \frac{n^{-2a}}{2} + \frac{n^{3a}}{3} - \ldots\right)\right\} \\
            &\sim n^{an} \exp(n^{1-a} - n^{1-2a}/2 + \cdots)
        \end{align*}
        where the term in the exponential terminates after the last nonnegative exponent of $n$.
        \item Completely analogous without factoring out $n^{an}$:
        \[
            (1+n^{-a})^n \sim \exp(n^{1-a} - n^{1-2a}/2 + \cdots)
        \]
        \item \[
            (1+n^{-a}\log n)^n = \exp(n\log(1+n^{-a}\log n)) \estep{\eqref{eq:power_loggeom}} \exp(n (n^{-a}\log n + \cdots)) \sim 1
        \]
        because $\log n \in \mathcal{O}(n^\epsilon)$ for any $\epsilon > 0$ and therefore $n^{-a + \epsilon}\in o(n^{-1})$ (note that $1<a<2$) such that $n^{1-a+\epsilon}$ approaches $0$.
    \end{enumerate}
\end{solution}

\begin{exercise}
    The purpose of this exercise is to find the asymptotic behavior of the number $a_n$ of permutations of $n$ letters whose cycles are all of length $\leq 3$, by using Hayman's method and the Lagrange Inversion Formula. (The use of a symbolic manipulation package on a computer is recommended for this exercise, in order to help out with some fairly tedious calculations with power series that will be necessary.) The egf of $\{a_n\}$ is 
    \[
        f(z) = \exp\left\{z+\frac{z^2}{2}+\frac{z^3}{3} \right\}
    \]
    \begin{enumerate}[label=(\alph*)]
        \item Show that $f$ is admissible in the plane.
        \item Because $r_n$ in this case satisfies a \emph{cubic} equation rather than a \emph{quadratic}, as in the example in this text, we will use the LIF to find the root and its powers with sufficient precision. Show that if we write
        \[
            u=1/r_n;\ t=n^{-1/3};\ \phi(u) = (1+u+u^2)^{1/3}
        \]
        then $u$ satisfies the equation $u=t\phi(u)$, which is in the desired form for LIF.
        \item Use the LIF to show that the root $r_n$ has the asymptotic expansion
        \[
            \frac{1}{r_n} = \frac{1}{n^{1/3}} + \frac{1}{3}\frac{1}{n^{2/3}}+\frac{1}{3}\frac{1}{n}+\frac{8}{81}\frac{1}{n^{4/3}}+ \mathcal{O}(n^{-5/3})
        \]
        \item Explain why the number of terms that were retained in part (c) is the minimum number that can be retained and still get the first term of the asymptotic expansion of $a_n$ with this method.
        \item Show that 
        \[
            \frac{1}{r_n^n} \sim n^{-\frac{n}{3}}\left\{\frac{1}{3}n^{2/3}+\frac{5}{18}n^{1/3} \right\}
        \]
        \item Show that 
        \[
            b(r_n) \sim 3n
        \]
        \item Show that\footnote{This is wrong in the book where it states $-\frac{29}{162}$ for the last term. See \url{https://oeis.org/A057693}} 
        \[
            f(r_n) \sim \exp\left\{ \frac{1}{3}n+\frac{1}{6}n^{2/3} +\frac{5}{9}n^{1/3}-\frac{5}{18}\right\}
        \]
        \item Combine the results of part (e), (f), (g) to show that the number of permutations of $n$ letters that have no cycles of length $> 3$ is 
        \[
            a_n \sim \frac{n^{\frac{2n}{3}}}{\sqrt{3}}\exp\left\{-\frac{2n}{3} + \frac{1}{2} n^{2/3}+\frac{5}{6}n^{1/3}-\frac{5}{18}\right\}
        \]
    \end{enumerate}
\end{exercise}
\begin{solution}
    \begin{enumerate}[label=(\alph*)]
        \item By the result of Exercise \ref{ex:5-6}, $\coeff{z^n}f(z)>0$ for all sufficiently large $n$ with $f(z) = e^{P(z)}$ which is a sufficient condition for being admissible in the plane.
        \item Calculating the auxiliary function first:
        \[
            a(r) = r\frac{f'(r)}{f(r)} = r+r^2+r^3
        \]
        Calculating $t\phi(u)$ with the given $t$ and $\phi(u)$:
        \[
            t\phi(u) = n^{-1/3}(1+u+u^2)^{1/3}
        \]
        Filling in $u=\frac{1}{r_n}$:
        \[
            t\phi(u) = n^{-1/3}\left(1+\frac{1}{r_n} + \frac{1}{r_n^2}\right)^{1/3} = n^{-1/3} \frac{1}{r_n}(r_n^3+r_n^2+r_n)^{1/3}
        \]
        Because $r_n$ is a root of $a(r_n) = n$:
        \[
            t\phi(u) = \frac{1}{r_n} n^{-1/3}n^{1/3} = \frac{1}{r_n} = u
        \]
        such that we obtain the desired form for Theorem~\ref{thm:lif}.
        \item Applying Theorem~\ref{thm:lif} to find the asymptotic expansion of $r_n^{-1} = u$ with $f(u) = u$:
        \begin{align*}
            \coeff{\frac{1}{n^{k/3}}} \frac{1}{r_n} &= \frac{1}{k} \coeff{u^{k-1}} (1+u+u^2)^{k/3}
        \end{align*}
        Using the binomial theorem twice:
        \begin{align*}
            \coeff{\frac{1}{n^{k/3}}} \frac{1}{r_n} \estepalign{\eqref{eq:binom_num}} \frac{1}{k} \coeff{u^{k-1}} \infsum[m] \binom{k/3}{m} (u+u^2)^m \\
            \estepalign{\eqref{eq:binom_num2}}  \frac{1}{k} \coeff{u^{k-1}} \infsum[m] \binom{k/3}{m} \infsum[l] \binom{m}{l} u^{m-l}u^{2l} \\
            &= \frac{1}{k} \coeff{u^{k-1}} \infsum[m] \infsum[l] \binom{k/3}{m} \binom{m}{l} u^{m+l} \\
            &= \frac{1}{k} \infsum[l] \binom{k/3}{k-1-l}\binom{k-1-l}{l}
        \end{align*}
        For the first four, we find:
        \begin{alignat*}{3}
            \coeff{\frac{1}{n^{1/3}}} \frac{1}{r_n} &= 1 
            &&\qquad \coeff{\frac{1}{n^{2/3}}} \frac{1}{r_n} = \frac{1}{3} \\
            \coeff{\frac{1}{n^{3/3}}} \frac{1}{r_n} &= \frac{1}{3} 
            &&\qquad \coeff{\frac{1}{n^{4/3}}} \frac{1}{r_n} = \frac{8}{81}
        \end{alignat*}
        thereby giving the desired result:
        \[
            \frac{1}{r_n} = \frac{1}{n^{1/3}} + \frac{1}{3}\frac{1}{n^{2/3}}+\frac{1}{3}\frac{1}{n}+\frac{8}{81}\frac{1}{n^{4/3}}+ \mathcal{O}(n^{-5/3})
        \]
        \item In Hayman's method, we want the asymptotic expansion for $\frac{1}{r_n^n}$. After factoring out $n^{-1/3}$ from the previous part, we obtain:
        \[
            \frac{1}{r_n} = \frac{1}{n^{1/3}} \left(1 + \frac{1}{3}\frac{1}{n^{1/3}}+\frac{1}{3}\frac{1}{n^{2/3}}+\frac{8}{81}\frac{1}{n}+ \mathcal{O}(n^{-4/3})\right)
        \]
        The second term will then be handled by doing an expansion as in Exercise~\ref{ex:5-7} and all terms which are not $o(n^{-1})$ inside need to be kept such that the first term of the asymptotic expansion is still correct.

        Similarly, for $f(r_n)$, we will need the constant term of $r_n^3$ which also needs terms up to this order (see later).
        \item \begin{align*}
            \frac{1}{r_n^n} &= n^{-n/3} \left(1 + \frac{1}{3}\frac{1}{n^{1/3}}+\frac{1}{3}\frac{1}{n^{2/3}}+\frac{8}{81}\frac{1}{n}+ \mathcal{O}(n^{-4/3})\right)^n  \\
            &=  n^{-n/3} \exp\left\{n \log\left(1 + \frac{1}{3}\frac{1}{n^{1/3}}+\frac{1}{3}\frac{1}{n^{2/3}}+\frac{8}{81}\frac{1}{n}+ \mathcal{O}(n^{-4/3})\right)\right\} \\
            &\overset{\eqref{eq:power_loggeom}}{\sim} n^{-n/3} \exp\left\{n\left( \frac{1}{3}\frac{1}{n^{1/3}} + \frac{1}{3}\frac{1}{n^{2/3}} + \frac{8}{81}\frac{1}{n} - \frac{1}{18} \frac{1}{n^{2/3}} - \frac{1}{9}\frac{1}{n} + \frac{1}{81}\frac{1}{n}\right) \right\} \\
            &\sim n^{-n/3} \exp\left\{n\left(\frac{1}{3}\frac{1}{n^{1/3}}+\frac{5}{18}\frac{1}{n^{2/3}}\right) \right\} \\
            &\sim n^{-n/3} \exp\left\{ \frac{1}{3}n^{2/3} + \frac{5}{18}n^{1/3}\right\}  \quad (n\to \infty)
        \end{align*}
        \item Calculating the auxiliary function $b(r)$:
        \[
            b(r) = ra'(r) = r + 2r^2 + 3r^3
        \]
        we then have:
        \[
            b(r_n) = r_n + 2r_n^2 + 3r_n^3 = n + r_n^2 +2r_n^3
        \]
        because $a(r_n) = r_n+r_n^2+r_n^3 = n$. The asmptotic expansion of $r_n$ is:
        \begin{align*}
            r_n &= n^{1/3} \frac{1}{1 + \frac{1}{3}\frac{1}{n^{1/3}}+\frac{1}{3}\frac{1}{n^{2/3}}+\frac{8}{81}\frac{1}{n} + \mathcal{O}(n^{-4/3})} \\
            &= n^{1/3} \left(1 - \frac{1}{3}\frac{1}{n^{1/3}} - \frac{1}{3}\frac{1}{n^{2/3}} - \frac{8}{81}\frac{1}{n} + \frac{1}{9}\frac{1}{n^{2/3}} + \frac{5}{27}\frac{1}{n} + \mathcal{O}(n^{-4/3})\right) \\
            &= n^{1/3} \left(1 - \frac{1}{3}\frac{1}{n^{1/3}} - \frac{2}{9}\frac{1}{n^{2/3}} +\frac{7}{81}\frac{1}{n} + \mathcal{O}(n^{-4/3}) \right) \\
            &= n^{1/3} - \frac{1}{3} - \frac{2}{9}\frac{1}{n^{1/3}} + \frac{7}{81}\frac{1}{n^{2/3}} + \mathcal{O}(n^{-1}) \\
            & \sim n^{1/3} \quad (n\to \infty)
        \end{align*}
        Plugging this in the previous expression, we obtain:
        \[
            b(r_n) = n + r_n^2 + 2r_n^3 \sim n + 2r_n^3 \sim 3n \quad (n\to \infty)
        \]
        \item First calculating $r_n^2$ and $r_n^3$ up to their constant terms:
        \begin{align*}
            r_n^2 &= \left(n^{1/3} - \frac{1}{3} - \frac{2}{9}\frac{1}{n^{1/3}} + \frac{7}{81}\frac{1}{n^{2/3}} + \mathcal{O}(n^{-1})\right)^2 \\
            &= n^{2/3} - \frac{1}{3}n^{1/3} - \frac{2}{9} + \frac{7}{81}\frac{1}{n^{1/3}} - \frac{1}{3}n^{1/3} + \frac{1}{9} \\
            &\mspace{50mu}+ \frac{2}{27}\frac{1}{n^{1/3}} - \frac{2}{9} + \frac{2}{27}\frac{1}{n^{1/3}} +\frac{7}{81}\frac{1}{n^{1/3}} + \mathcal{O}(n^{-2/3})\\
            &= n^{2/3} - \frac{2}{3}n^{1/3} - \frac{1}{3} + \frac{26}{81}\frac{1}{n^{1/3}}+ \mathcal{O}(n^{-2/3}) \\
            &= n^{2/3} - \frac{2}{3}n^{1/3} - \frac{1}{3} + \mathcal{O}(n^{-1/3})
        \end{align*}
        \begin{align*}
            r_n^3 &=  \left(n^{1/3} - \frac{1}{3} - \frac{2}{9}\frac{1}{n^{1/3}} + \frac{7}{81}\frac{1}{n^{2/3}} + \mathcal{O}(n^{-1})\right)^3 \\
            &= \left(n^{2/3} - \frac{2}{3}n^{1/3} - \frac{1}{3} + \frac{26}{81}\frac{1}{n^{1/3}}+ \mathcal{O}(n^{-2/3})\right)r_n \\
            &= n - \frac{1}{3}n^{2/3} - \frac{2}{9}n^{1/3} + \frac{7}{81} - \frac{2}{3}n^{2/3} \\
            &\mspace{50mu}+ \frac{2}{9}n^{1/3} + \frac{4}{27} - \frac{1}{3}n^{1/3} + \frac{1}{9} + \frac{26}{81} + \mathcal{O}(n^{-1/3}) \\
            &= n - n^{2/3} - \frac{1}{3}n^{1/3}  + \frac{2}{3} + \mathcal{O}(n^{-1/3})
        \end{align*}
        Plugging these expressions into $f(r_n)$:
        \begin{align*}
            f(r_n) &= \exp\left\{r_n + \frac{r_n^2}{2} + \frac{r_n^3}{3} \right\} \\
            &=\exp\left\{n - \frac{r_n^2}{2} - \frac{2r_n^3}{3} \right\} \\
            &= \exp\left\{n - \frac{1}{2}n^{2/3} + \frac{1}{3}n^{1/3} + \frac{1}{6} - \frac{2}{3}n \right. \\
            &\mspace{90mu} \left.+ \frac{2}{3}n^{2/3} + \frac{2}{9}n^{1/3} - \frac{4}{9} + \mathcal{O}(n^{-1/3})\right\} \\
            &= \exp\left\{\frac{1}{3}n + \frac{1}{6}n^{2/3} + \frac{5}{9}n^{1/3} - \frac{5}{18}\right\}
        \end{align*}
        \item From Hayman's method (Theorem~\ref{thm:hayman}), we immediately find that:
        \begin{align*}
            \frac{a_n}{n!} &\sim \frac{n^{-n/3} \exp\left\{ \frac{1}{3}n^{2/3} + \frac{5}{18}n^{1/3}\right\}\exp\left\{\frac{1}{3}n + \frac{1}{6}n^{2/3} + \frac{5}{9}n^{1/3} - \frac{5}{18} \right\}}{\sqrt{6\pi n}} \\
            &\sim \frac{n^{-n/3}}{\sqrt{6\pi n}} \exp\left\{\frac{1}{3}n + \frac{1}{2}n^{2/3}+\frac{5}{6}n^{1/3}-\frac{5}{18}\right\}
        \end{align*}
        With Stirling's formula, $n! \sim \frac{n^n \sqrt{2n\pi}}{e^n}$, the number of permutations of $n$ letters that have no cycles of length $> 3$ is:
        \[
            a_n \sim \frac{n^{\frac{2n}{3}}}{\sqrt{3}} \exp\left\{-\frac{2n}{3} + \frac{1}{2}n^{2/3} + \frac{5}{6}n^{1/3} - \frac{5}{18} \right\}
        \]
    \end{enumerate}
\end{solution}

\begin{exercise}
    Derive the power series expansion 
    \[
        \left(\frac{1-\sqrt{1-4x}}{2x}\right)^k = \bsum[n] \frac{k(2n+k-1)!}{n!(n+k)!}x^n \quad (k\geq 1)
    \]
\end{exercise}
\begin{solution}
    Note that:
    \[
        \frac{1-2x - \sqrt{1-4x}}{2x} = \frac{1-\sqrt{1-4x}}{2x} - 1
    \]
    is the root of the quadratic $u = x(1+2u + u^2) = x(1+u)^2$ which vanishes at $x=0$. The asked power series is the expansion of $f(u) = (1+u)^k$ where $u=x(1+u)^2$. Applying Theorem~\ref{thm:lif} gives:
    \begin{align*}
        \coeff{x^n} \left(\frac{1-\sqrt{1-4x}}{2x}\right)^k &= \frac{1}{n}\coeff{u^{n-1}} \left\{k(1+u)^{2n+k-1}\right\} \\
       \estepalign{\eqref{eq:binom_num}} \frac{k}{n} \coeff{u^{n-1}} \infsum[m] \binom{2n+k-1}{m}u^m \\
       &= \frac{k}{n} \binom{2n+k-1}{n-1} \\
       &= \frac{k(2n+k-1)!}{n!(n+k)!}
    \end{align*}
    Multiplying both sides by $x^n$ and summing over $n\geq 0$ gives the desired result:
    \[
        \left(\frac{1-\sqrt{1-4x}}{2x}\right)^k = \bsum[n] \frac{k(2n+k-1)!}{n!(n+k)!}x^n \quad (k\geq 1)
    \]
\end{solution}

\begin{exercise}
    In this exercise, $\sigma(n,k)$ is the number of involutions of $n$ letters that have exactly $k$ cycles, and $t_n=\sum_k \sigma(n,k)$ is the number of involutions of $n$ letters.
    \begin{enumerate}[label=(\alph*)]
        \item Show that 
        \[
            \sum_{n,k} \frac{\sigma(n,k)}{n!} x^n y^k = e^{y(x+ \frac{1}{2}x^2)}
        \]
        \item Hence find the formula 
        \[
            \sigma(n,k) = \frac{n!}{(n-k)!(2k-n)!2^{n-k}}
        \]
        for $\sigma(n,k)$.
        \item Using the results of part (a) and Exercise \ref{ex:3-5}, show that the \emph{average} number of cycles in an involution of $n$ letters is exactly
        \[
            \frac{n}{2}\left\{1+\frac{t_{n-1}}{t_n} \right\}
        \]
        \item Using 
        \[
            t_n \sim \frac{1}{\sqrt{2}}n^{n/2}\exp\left(-\frac{n}{2}+\sqrt{n}-\frac{1}{4}\right)
        \]
        show that the average number of cycles in an involution of $n$ letters is 
        \[
            = \frac{n}{2} + \frac{1}{2}\sqrt{n}(1+o(1)) \qquad (n\to \infty)
        \]
    \end{enumerate}
\end{exercise}
\begin{solution}
    \begin{enumerate}[label=(\alph*)]
        \item Involutions can have cycle lengths of $1$ or $2$ only. Consider the usual exponential family where the cards are cycles of permutations. Only cycles lengths of $1$ and $2$ are permitted such that the deck enumerator is:
        \[
            \mathcal{D}(x) = \frac{(1-1)! x}{1!} + \frac{(2-1)!x^2}{2!} = x + \frac{x^2}{2}
        \]
        The hands of weight $n$ and $k$ cards are then involutions of $n$ letters with $k$ cycles. By the exponential formula, the hand enumerator is:
        \[
            \mathcal{H}(x,y) = \bsum[n]\bsum[k] \sigma(n,k) \frac{x^n}{n!} y^k = e^{y(x+\frac{1}{2}x^2)}
        \]
        \item Extracting the coefficient of $\frac{x^n}{n!}y^k$:
        \begin{align*}
            \sigma(n,k) &= \coeff{\frac{x^n}{n!}y^k} \mathcal{H}(x,y) \estep{\eqref{eq:power_exp}} \coeff{\frac{x^n}{n!}y^k} \bsum[l] \left(x+\frac{1}{2}x^2\right)^l \frac{y^l}{l!} \\
            &= \coeff{\frac{x^n}{n!}} \left(x+\frac{1}{2}x^2\right)^k\frac{1}{k!} \estep{\eqref{eq:binom_num}} \coeff{\frac{x^{n-k}}{n!}}\frac{1}{k!} \infsum[m] \binom{k}{m} 2^{-m}x^m  \\
            &= \frac{n!}{k!}\binom{k}{n-k} 2^{-(n-k)} = \frac{n!}{(n-k)!(2k-n)!2^{n-k}}
        \end{align*}
        \item Differentiating $\mathcal{H}(x,y)$ with respect to $y$ and setting $y=1$ gives the following egf:
        \[
            \bsum[n] \frac{x^n}{n!} \bsum[k] k\sigma(n,k) = \left(x+\frac{1}{2}x^2\right)e^{(x+\frac{1}{2}x^2)}
        \]
        Extracting coefficients of $\frac{x^n}{n!}$ on both sides:
        \[
            \bsum[k] k\sigma(n,k) = \coeff{\frac{x^n}{n!}} xe^{x+\frac{1}{2}x^2} + \frac{1}{2}\coeff{\frac{x^n}{n!}} x^2e^{x+\frac{1}{2}x^2}
        \]
        From Exercise~\ref{ex:3-5}, we know that the egf of $t_n$ is $e^{x+\frac{1}{2}x^2}$ such that:
        \begin{align*}
            \bsum[k] k\sigma(n,k) &= \coeff{\frac{x^n}{n!}} \bsum[m] t_m \frac{x^{m+1}}{m!} + \frac{1}{2}\coeff{\frac{x^n}{n!}} \bsum[m] t_m \frac{x^{m+2}}{m!} \\
            &= nt_{n-1} + \frac{1}{2}n(n-1)t_{n-2}
        \end{align*}
        The average number of cycles results from dividing by the total number of involutions:
        \[
            \frac{1}{t_n}\bsum[k] k\sigma(n,k) = n\frac{t_{n-1}}{t_n} + \frac{1}{2t_n}n(n-1)t_{n-2}
        \]
        Using the recurrence relation from Exercise~\ref{ex:3-5} to eliminate $n(n-1)t_{n-2}$:
        \[
            \frac{1}{t_n}\bsum[k] k\sigma(n,k) = n\left(\frac{t_{n-1}}{t_n} + \frac{t_n - t_{n-1}}{2t_n}\right) = \frac{n}{2}\left(1+\frac{t_{n-1}}{t_n} \right)
        \]
        \item Plugging the asymptotic expansion in the result of the previous part yields:
        \begin{align*}
            \frac{1}{t_n}\bsum[k] k\sigma(n,k) &\sim \frac{n}{2} + \frac{n}{2}\frac{\frac{1}{\sqrt{2}} (n-1)^{(n-1)/2}\exp\left(-\frac{(n-1)}{2} + \sqrt{n-1} - \frac{1}{4}\right)}{\frac{1}{\sqrt{2}} n^{n/2} \exp\left(-\frac{n}{2}+\sqrt{n} - \frac{1}{4}\right)} \\
            &\sim  \frac{n}{2} + \frac{n}{2}n^{(n-1)/2 - n/2} \left(1-\frac{1}{n}\right)^{\frac{n-1}{2}} e^{\left(\frac{1}{2} + \sqrt{n-1} - \sqrt{n} \right)} \\
            &\sim  \frac{n}{2} + \frac{\sqrt{n}}{2}  \frac{\sqrt{e}}{\sqrt{e}} \frac{e^{\sqrt{n-1}}}{e^{\sqrt{n}}} \\
            &\sim  \frac{n}{2} + \frac{1}{2} \sqrt{n}(1+o(1)) \qquad (n\to \infty)
        \end{align*}
    \end{enumerate}
\end{solution}